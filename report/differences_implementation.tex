In contrast to the definition in previous reports, we made two changes in the final implementation.

The first is that we did not implement the coalition formation as planned. Instead, there are multiple winners of the contract net. For instance, if there is more metal on one cell than a single digger agent is able to carry, the digger coordinator agent can decide to accept more than one proposal for the contract net, leading to more than one winner to carry out the task. Thus, two diggers share the task of digging the field, however they are not aware of the other digger agent assigned to the same task (in contrast with the structure of coalitions).

We decided not to implement the coalition formation, as it doesn't bring many advantages.  It's not necessary for the digger agents to know about the other digger, since they don't need to communicate with each other, but they can solve the task independently. Also, the calculation of the coalition values can be very memory-consuming. The solution we implemented avoids this costly calculation and yet there can still be multiple digger agents sharing a task.

The second difference in the implementation is the way the agents deal with path conflicts. Initially we wanted to calculate a new path, for instance if there is another digger agent excavating in the path cell. Now, we decided that the agent should wait for the digging agent to finish, rather than looking for the second best path. We decided this because we saw that in most situations it is unlikely that taking the second best path is faster than waiting for the other agent to finish, and it also enables the system to avoid extra computation.


