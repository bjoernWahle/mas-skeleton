When implementing the real system we faced some problems with the programming environment and we could not implement everything as planned. In this section, we will explain these differences that could may be added in future works. In contrast to the definition in previous reports, we made two changes in the final implementation: the coalition formation differs a bit from the planned one and the path conflicts are just avoided. 

\subsection{Coalition formation}

The first difference from the planned implementation is that we did not implement the coalition formation as we originally planned. We planned that the diggers would form coalitions between them autonomously, without supervision of the digger coordinator and deciding themselves whether they need it or not. Instead, we did a different and much simpler thing, the digger coordinator creates these coalitions if it is necessary and the diggers cooperate to dig the metal in an implicit way.  For instance, if there is more metal on one cell than a single digger agent is able to carry, the digger coordinator agent can decide to accept more than one proposal for the contract net, leading to more than one winner to carry out the task. Thus, two diggers share the task of digging the field, however they are not aware of the other digger agent assigned to the same task (in contrast with the structure of coalitions).

We decided not to implement the coalition formation, as it doesn't bring many advantages but increases a lot the complexity.  It's not necessary for the digger agents to know about the other digger, since they don't need to communicate with each other, but they can solve the task independently. Also, the calculation of the coalition values can be very memory-consuming. The solution we implemented avoids this costly calculation and yet there can still be multiple digger agents sharing a task.

\subsection{Dealing with path conflicts}

The second difference in the implementation is the way the agents deal with path conflicts. This is when an agent finds a digger working in the middle of the way, it cannot go through the same cell and must wait or find another path. 

Initially we wanted to solve this problems by calculating a new path, modifying a bit the planned route to avoid the conflict. However, the algorithm to calculate the optimal path is complex and very time consuming. Changing it to deal with this conflicts was complex and lead to more complex algorithms. 

Instead, we decided that the agent should wait for the digging agent to finish, rather than looking for the second best path. We decided this because we saw that in most situations it is unlikely that taking the second best path is faster than waiting for the other agent to finish, and it also enables the system to avoid extra computation.


